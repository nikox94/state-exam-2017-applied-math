\documentclass[11pt]{article}
\usepackage[a4paper,includeheadfoot,margin=2.54cm]{geometry}
\usepackage[T2A]{fontenc}
\usepackage[utf8]{inputenc}
\usepackage[bulgarian]{babel}
\usepackage[unicode=true]{hyperref}
\usepackage{breakurl}
\usepackage{indentfirst}
\usepackage{pgf, tikz}
\usepackage{amsmath}
\usepackage{amsthm}
\usepackage{amssymb}
\usepackage{mathtools}
\usepackage{esvect}
\usepackage{breqn}

\numberwithin{equation}{section}
\numberwithin{figure}{section}
\numberwithin{table}{section}
  \theoremstyle{plain}
  \newtheorem{thm}{\protect\theoremname}[section]
  \theoremstyle{definition}
  \newtheorem{defn}[thm]{\protect\definitionname}
  \theoremstyle{remark}
  \newtheorem*{notation*}{\protect\notationname}
  \theoremstyle{definition}
  \newtheorem*{example*}{\protect\examplename}
  \theoremstyle{remark}
  \newtheorem*{note*}{\protect\notename}
  \theoremstyle{plain}
  \newtheorem{lem}[thm]{\protect\lemmaname}
  \theoremstyle{definition}
  \newtheorem*{defn*}{\protect\definitionname}
  \theoremstyle{definition}
  \newtheorem{example}[thm]{\protect\examplename}
  \theoremstyle{plain}
  \newtheorem{cor}[thm]{\protect\corollaryname}
  \theoremstyle{plain}
  \newtheorem{prop}[thm]{\protect\propositionname}
  \theoremstyle{plain}
  \newtheorem*{prop*}{\protect\propositionname}
  \theoremstyle{definition}
  \newtheorem{xca}[thm]{\protect\exercisename}
  \newcommand\thmsname{\protect\theoremname}
%  \newcommand\nm@thmtype{theorem}
  \theoremstyle{plain}
  \newtheorem*{namedtheorem}{\thmsname}
  \newenvironment{namedthm}[1][Undefined Theorem Name]{
   \ifx{#1}{Undefined Theorem Name}\renewcommand\nm@thmtype{theorem*}
   \else\renewcommand\thmsname{#1}\renewcommand\nm@thmtype{namedtheorem}
   \fi
   \begin{\nm@thmtype}}
   {\end{\nm@thmtype}}

  \providecommand{\corollaryname}{Следствие}
  \providecommand{\definitionname}{Дефиниция}
  \providecommand{\examplename}{Пример}
  \providecommand{\exercisename}{Упражнение}
  \providecommand{\lemmaname}{Лема}
  \providecommand{\notationname}{Нотация}
  \providecommand{\notename}{Забележка}
  \providecommand{\propositionname}{Твърдение}
  \providecommand{\remarkname}{Забележка}
  \providecommand{\theoremname}{Теорема}

\DeclarePairedDelimiter\norm{\lVert}{\rVert}
\renewcommand*{\Vec}[1]{\mathbf{#1}}
\newcommand*{\Z}{\Vec{0}}
\newcommand*{\B}{\mathcal{B}}
\newcommand*{\R}{\mathbb{R}}
\newcommand*{\N}{\mathbb{N}}
\newcommand*{\Q}{\mathbb{Q}}
\newcommand*{\p}{\partial}

\title{Разписани теми за държавен изпит, приложна математика, ФМИ на СУ, 2017г.}
\author{Никола Юруков}
\date{\today}


\begin{document}

\maketitle

\clearpage

\tableofcontents

\clearpage

Задачи са за теми: 1-8 и 10-19

\section{Уравнения на права и равнина. Формули за разстояния и ъгли. Криви от втора степен.}

\textbf{Векторни и параметрични (скаларни) уравнения на права и равнина. Общо уравнение на права в равнината. Декартово уравнение. Взаимно положение на две прави. Нормално уравнение на права. Разстояние от точка до права. Ъгъл между прави.}

\textbf{Общо уравнение на равнина. Взаимно положение на две равнини. Нормално уравнение на равнина. Разстояние от точка до равнина.}

\textbf{Уравнение на окръжност. Канонични уравнения на елипса, хипербола и парабола. Фокални свойства на елипса, хипербола и парабола.}

\section{Алгебрическа затвореност на полето на комплексните числа. Следствия. Формули на Виет.}

\section{Симетрични оператори в крайномерни евклидови пространства. Основни свойства. Теорема за
диагонализация.}

\hrulefill

\textbf{Изисквания по конспект.} Симетричен оператор – определение, матрица спрямо ортонормиран базис. Всички характеристични
корени на симетричен оператор са реални числа; всеки два собствени вектора, съответстващи на различни
собствени стойности, са ортогонални помежду си; съществува ортонормиран базис на пространството, в който
матрицата на симетричен оператор е диагонална.
Примерна задача: За даден симетричен оператор да се намерят ортонормиран базис на пространството,
в който матрицата му е диагонална, както и самата матрица.
Литература: [25]

\hrulefill

Нека $V$ е $n$-мерно евклидово пространство.


\begin{defn}
\textbf{Симетричен оператор.}

Нека $\varphi \in HomV$. Симетричен е ако $\forall u,v \in V : (\varphi(u),v) = (u,\varphi(v))$
\end{defn}

..Редуцираме твърдението само до базисни вектори. $(\varphi(e_i),e_j) = (e_i,\varphi(e_j))$


\begin{defn}
\textbf{Симетрична матрица.}

Една матрица е симетрична ако $A^T = A$ или $a_{ij} = a_{ji} \forall i,j \in I$
\end{defn}

\begin{prop}
\textbf{Свойства на симетричните матрици}
\begin{enumerate}
\item Подпространство на симетричните матрици
\item Ако $\exists A^{-1} \Rightarrow A^{-1}$ също е симетрична + доказателство
\item $A$ и $B$ комутират $\Rightarrow AB$ е симетрична + доказателство
\end{enumerate}
\end{prop}


\begin{prop}
\textbf{Оператор<->матрица.}
$\varphi$ е симетричен оператор тогава и само тогава когато $A$ е симетрична матрица.
\end{prop}
\begin{proof}
Нека $e_1, e_2, ... e_n$ е ортонормиран базис на $V$, $\varphi \in HomV$ и $A = (a_{ij})$ е матрицата на $\varphi$ в този базис.

Разпадаме един вектор на компонентните му базисни, после като разгледаме скаларното произведение излиза веднага твърдението.
\end{proof}

\begin{prop}
\textbf{$\lambda$ са реални числа.}
Характеристичните корени на симетрична матрица са реални числа
\end{prop}
\begin{proof}
\textbf{Гадно е.}
Разглеждат се уравнения за коерните на характеристичния полином и от там се извеждат спрегнатости и реалност.
\end{proof}

\begin{prop}
\textbf{$\lambda$ са реални числа за оператор.}
\end{prop}

\begin{prop}
\textbf{Ортогонални базисни вектори.}
\end{prop}

\begin{prop}
\textbf{Диагонализируем оператор. ГАААДНО.}
\end{prop}

\begin{prop}
\textbf{Диагонализируема матрица.}
\end{prop}

\hrulefill

Задачи



\end{document} \grid
